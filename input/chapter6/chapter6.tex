\chapter{Spellcasting}

\begin{figure}[h]
	\includegraphics[width=\textwidth]{flamespell.png}
\end{figure}

Magic in the Elder Scrolls generally refers to the manipulation of magicka (magical energy) to alter the world in some way. Individual castings of this to achieve effects are known as spells. It can come from many sources, such as starlight, the Divines or Daedric Princes. All the races of Tamriel have some magical aptitude, though those of elvish blood are more naturally gifted.\\

There are many different forms of magic in Elder Scrolls and they have all been handled differently across the franchise. This game's magic system is based primarily on that from Elder Scrolls IV: Oblivion. Player characters are able to draw on the magicka in their bodies to cast spells, which are governed by the schools of Alteration, Conjuration, Destruction, Illusion, Mysticism and Restoration. Alchemy is also a magic skill in that it involves the manipulation of magical properties in alchemical reagents to create potions and poisons; it will be covered in the next chapter.\\

Spellcasting is somewhat similar to combat techniques in that it costs points, has the potential to give you greater damage and special effects, and gives you access to gradually more powerful abilities as your skills increase. However, it is different from techniques in that it uses your magicka instead of stamina; spells that deal damage also do not get attribute- or skill-based bonus damage. Spells have static values for their magnitude instead of rolls and so are often more reliable than weapon attacks. Also, increased skill in one of the magic schools reduces the cost of spells in that school, meaning you will have access to far more powerful spells as you level up.

\section{Spellcasting Formulas}
For convenience, here are the max magicka, magicka regeneration and spell range formulas again:

\begin{itemize}
	\item $\text{Magicka}=2*\text{Intelligence}$
	\item $\text{Range}=2*\text{Willpower}$
	\item $\text{Regeneration}=(5+0.1*\text{Willpower})\%$ of max magicka per round
\end{itemize}

\section{Spell Properties}
Each spell costs an action to cast in combat and comes with the following properties:

\begin{itemize}
	\item \textbf{Effect}: What does the spell do?
	\item \textbf{Target}: How the spell's effect is applied (at range, on touch, or on self).
	\item \textbf{Base Magicka Cost}: How much magicka you need to spend to cast this spell.
	\item \textbf{Magnitude}: How big the spell's effect is (e.g. damage dealt, health restored).
	\item \textbf{Duration}: How long the spell's effect lasts.
	\item \textbf{Area}: The radius in which spell effects apply.
	\item \textbf{School}: Which skill is used with the spell.
	\item \textbf{Skill Level}: What skill level is required to cast the spell (Novice, Apprentice, Journeyman, Expert or Master).
	\item \textbf{Action}: Does the spell cost an action, or can it be cast as a reaction or bonus action?
\end{itemize}

Note that not all spells have each property. Most spells do not have an area effect but rather act on a single target, for example. Some spells have multiple effects, each with a different duration and sometimes different targeting styles.\\

Your skill affects the magicka cost of a spell according to the following formula:

\begin{center}
$\text{Cost}=\text{Base Cost}*(1.4-0.012*\text{Skill})$
\end{center}

Outside of battle or other time-critical situations, magicka costs will be ignored because it is assumed you have enough time to regenerate your magicka reserves. If you are ambushed or otherwise threatened immediately after casting a spell outside of battle, the GM may rule that you must deduct the spell cost from your current magicka in the first round. Also, casting outside of combat still requires that you have enough max magicka to cast the spell, and skill requirements still apply.

\subsection{Spell Attacks}
If the target of your spell is unwilling, you will have to make a spell attack roll. Touch spells count as melee attacks, and ranged spells count as ranged attacks; both categories are affected by the bonus and penalty conditions that might normally affect an attack roll. In order to make the attack roll, make sure your target is within range and roll against the skill that governs the spell. If you are successful, the target is hit and the spell takes effect.\\

If you have a free hand, you may cast another spell of Apprentice level or lower as a bonus action. You must have the magicka to do so, and you must make a spell attack roll if applicable. All the normal rules for spellcasting apply.\\

Note that spells make noise when cast and so are difficult to use undetected. Journeymen of Illusion may cast any spell silently.

\section{Schools of Magic}
Mainstream culture in Tamriel recognizes six schools of magic, each concerned with different magical effects. Beginning the game with a school of magic as a major skill will grant you starting skills from that school. All classes with magic for at least one major skill begin the game with the following spells in addition to those granted by the schools:

\begin{itemize}
	\item \textbf{Flare} (\textit{11 Magicka, Ranged, Destruction Novice}): The target takes 6 points of fire damage and an additional 1d4 on its next turn.
	\item \textbf{Heal Minor Wounds} (\textit{14 Magicka, Self, Restoration Novice}): You recover 8 health.
\end{itemize}

In combat, casting a spell requires an action unless otherwise specified.

\subsection{Alteration}
The school of Alteration covers spells that manipulate the natural world and its physical properties. Alteration magic allows you to do things like resist physical and elemental attacks, change the weight of items, open locks, and breathe underwater. Some Alteration spells target opponents with negative effects, requiring a spell attack roll.\\

If Alteration is one of your major skills, you begin the game with the following spells:
\begin{itemize}
	\item \textbf{Open Very Easy Lock} (\textit{7 Magicka, Ranged, Novice}): Automatically opens a lock of Very Easy difficulty.
	\item \textbf{Protect} (\textit{10 Magicka, Self, Novice}): +1 to your current armor rating for 5 rounds. You may cast this spell as a bonus action.
\end{itemize}

On an extreme success, Alteration attacks may have greater magnitude or duration.

\subsection{Conjuration}
\begin{wrapfigure}{L}{0.25\textwidth}
	\includegraphics[width=\textwidth]{conjurer.png}
\end{wrapfigure}

The school of Conjuration is used primarily to summon things from Oblivion. It can be used to conjure daedra and undead as well as summon daedric weapons and armor for very brief periods of time. Summoned creatures will act independently, but are allied with you. They may respond to commands if they are intelligent. You can only have one summoned minion at once. If you conjure a new minion before the current one's time runs out, the current one is dismissed. If you attack your minion, it will become hostile.\\

If Conjuration is one of your major skills, you begin the game with the following spells:
\begin{itemize}
	\item \textbf{Summon Skeleton} (\textit{45 Magicka, Self, Apprentice}): Summons a Skeleton minion for 7 rounds.
	\item \textbf{Turn Undead} (\textit{11 Magicka, Ranged, Novice}): Undead up to level 3 must use their turn to move as far from you as possible as long as they can see you, and they cannot attack you. This effect lasts until the end of the target's third turn.
\end{itemize}

\subsection{Destruction}
The school of Destruction involves all spells that directly harm the body, possessions or abilities. Destruction spells can be used to project flames, ice or lightning as well as break armor, fatigue combatants, and much more. Destruction spells almost always require a spell attack roll.\\

If Destruction is one of your major skills, you begin the game with the following spells:
\begin{itemize}
	\item \textbf{Cold Touch} (\textit{23 Magicka, Touch, Novice}): The target takes 15 points of frost damage and 7 points of stamina damage.
	\item \textbf{Shocking Touch} (\textit{14 Magicka, Touch, Novice}): The target takes 10 points of shock damage and 5 points of magicka damage.
\end{itemize}

On an extreme success, elemental damage spells get a bonus effect. Fire spells set the target on fire, dealing an additional 10\% fire damage over the course of the target's next three turns (damage is taken at the beginning of each turn). Frost spells root the target in place with ice, preventing them from moving on their next turn unless they succeed on a hard Athletics check. Shock spells jump to a nearby target within range, dealing the same effect on the new target.\\

The following are special abilities you gain access to at certain skill levels (see the mastery perk chart, section 3.2):

\begin{itemize}
\item \textbf{Dual Casting}: If you have two free hands, you can dual cast any destruction spell. Dual casted spells require double the magicka cost, but they deal triple damage.

\item \textbf{Elemental Mastery}: Choose an element: fire, frost or lightning. From now on, all spells that deal that damage type gain the corresponding effect:

\begin{itemize}
	\item \textbf{Fire}: Enemies under 20\% health are inflicted with fear for 3 turns when struck by a fire spell.
	\item \textbf{Frost}: Enemies under 20\% health are paralyzed for 3 turns when struck by a frost spell.
	\item \textbf{Lightning}: Enemies under 15\% health are immediately turned into ash when struck by a shock spell.
\end{itemize}

You may only choose one element to master, and you choose it once you achieve the mastery perk. The effects of this perk do not take effect on enemies that are resistant to that element, defined as having a resistance greater than 0.
\end{itemize}

\subsection{Illusion}
Illusion spells manipulate the senses and emotions of their targets. Illusion is useful both in and out of battle with effects such as charms, invisibility, muffling, calming, frenzy and courage. Some illusions will require spell attack rolls. People usually do not like to be tricked with illusions, so using them in social situations will require subterfuge. Illusionists must be clever!\\

If Illusion is one of your major skills, you begin the game with the following spells:
\begin{itemize}
	\item \textbf{Soothing Touch} (\textit{13 Magicka, Touch, Novice}): Targets up to level 2 are calmed for 2 rounds. Calmed targets will not move or perform any actions. They are aware of your presence, and the effect will end if you attack them.
	\item \textbf{Starlight} (\textit{14 Magicka, Self, Novice}): You emanate bright light in a 20 ft radius and dim light in a 40 ft radius for 1 minute. You may cast this spell as a bonus action.
\end{itemize}

On an extreme success, Illusion attacks may have increased magnitude or duration.

\subsection{Mysticism}
The school of Mysticism is concerned with unraveling the mysteries of the world. Spells under this school are grouped less by categorical membership characteristics and more by tradition, as these spells were considered a single school by the Psijic Order, a group of mages who created them. Effects in this school allow you to detect living creatures, dispel, reflect or absorb magical effects, harvest souls for enchanting, and move distant objects. Some Mysticism spells target opponents and will require attack rolls.\\

If Mysticism is one of your major skills, you begin the game with the following spells:
\begin{itemize}
	\item \textbf{Minor Dispel} (\textit{22 Magicka, Self, Novice}): For all magical effects currently active on you, those that were imposed by a spell that cost no more than 40 magicka are immediately ended. Applies to beneficial and harmful spells alike. Does not apply to magical effects imposed by abilities, diseases, curses, enchantments, potions, poisons or scolls.
	\item \textbf{Minor Life Detection} (\textit{15 Magicka, Self, Novice}): For the next 12 seconds, all living creatures within 60 feet of you appear to give off a faint purple glow that you can see through solid objects. Only you can see this effect.
\end{itemize}

On an extreme success, Mysticism spells may have increased magnitude or duration.

\subsection{Restoration}
Restoration magic involves all spells which heal, rest, cure, restore or fortify a target. The effects of Restoration are many and varied, and they include things such as restore health, restore stamina, fortify health, stamina and magicka, fortify skills and attributes, cure poison and diseases and resist magicka. A few Restoration spells can act as attacks, though not many.\\

\begin{figure}[h]
	\includegraphics[width=\textwidth]{healing.png}
\end{figure}

If Restoration is one of your major skills, you begin the game with the following spell:
\begin{itemize}
	\item \textbf{Absorb Health} (\textit{12 Magicka, Touch, Novice}): Transfer 5 health from the target to you.
	\item \textbf{Minor Healing Hands} (\textit{32 Magicka, Touch, Apprentice}): Heals the target for 15 health.
\end{itemize}

On an extreme success, Restoration spells may have increased magnitude or duration.

\section{Learning New Spells}

In the Elder Scrolls, there is no limit on how many spells you can know. You can learn new spells by finding spell vendors in towns or in Mages Guild buildings, reading spell tomes, or even crafting your own spells if you can gain access to a rare spellmaking altar.

\subsection{Spellmaking}
If you have access to a spellmaking altar, you can make your own spells. Such altars are exceedingly rare; for example, the only spellmaking altars in Cyrodiil are found at the Arcane University of the Imperial City, which can only be accessed by Mages Guild affiliates who have undergone a certain level of training. Once you have acquired access, however, use the following rules to create a spell:

\begin{itemize}
	\item You get to choose the spell's name.
	\item To give the spell a certain effect, you must already know a spell with that effect. This includes race and birthsign powers.
	\item Multiple effects may be added to a single spell.
	\item For each effect, you choose the targeting type, magnitude, duration and area.
	\item Calculate the magicka cost using the following formula: $\text{Total Cost}=0.1*\text{Base Cost}*0.15*\text{Area}*\text{Duration}*\text{Magnitude}^{1.28}$
\end{itemize}

If the spell is ranged, multiply the result by 1.5 to get the final base cost. Duration is in rounds. Ask your GM for information on the base costs of effects or use the following link: \url{http://en.uesp.net/wiki/Oblivion:Spell_Effects}\\

Once you have these things determined, you must perform a 2-hour ritual to create the spell. The materials for the ritual cost a total of three times the magicka cost of the spell.
