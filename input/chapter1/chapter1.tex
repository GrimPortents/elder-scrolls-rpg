\chapter{Basic Gameplay}
Before you begin making a character, you should understand the basic mechanics underlying the game. To play, you will need four-sided, six-sided, eight-sided, ten-sided, twelve-sided, twenty-sided and percentile dice. You can get away with only having one of each, but having multiple may make rolls easier. Dice are abbreviated according to how many sides they have. A d4 has four sides; a d10 has ten sides. Easy enough? This abbreviation can also accept a coefficient. If you are told to roll 2d6, that means to roll two six-sided dice and add their outcomes.

\section{Roleplaying}
The Elder Scrolls Tabletop Role Playing Game is a game focused on roleplaying. It's right in the name! Though you will find this book full of numbers, multipliers and percentages, all these formulas only exist to support a narrative. That narrative is the primary manner in which you will play this game.\\

Scenes in this game will typically follow a simple pattern. First, the GM will describe the scene and what is present in it. Next, the players describe how their characters act in relation to that scene. Then the GM narrates the outcome of that interaction. Whenever there is uncertainty about what the outcome of an action will be, the GM will ask the player to make an appropriate roll. The roll will decide what happens. Examples of such situations include attacks and skill-based actions.

\section{Making a Roll}
Whenever you are called upon to make a roll to determine success, you will need to roll a d10 and a percentile die. The percentile die determines the tens place, and the d10 determines the ones place. For example, if you roll a 40 on the percentile die and a 3 on the d10, your roll is 43. Some ten-sided dice are numbered 1-10 while others are numbered 0-9; rolling a 0 or a 10 counts as a 0 for the ones place. If you roll a 00 on the percentile die and a 0 on the d10, that means your roll is a 100.\\

Whatever the outcome of your roll, you will usually be asked to compare it to the value of a certain skill. Having a Blade skill of 40, for example, means you have a 40\% chance of succeeding whenever you roll to determine success on attacks with bladed weapons. This is accomplished by comparing the outcome of your roll to that value. To succeed on a roll, the outcome of that roll must be less than or equal to the specified value.

\subsection{Levels of Success}
Rolls are not all pass or fail. This game features \textit{success levels}, which are very important for opposed skill checks and may result in special bonuses! The outcomes of a roll are as follows:

\begin{figure}[h]
\begin{tabular}{p{0.3\textwidth}p{0.7\textwidth}}
	\textbf{Fumble} & 96-100\\
	\textbf{Failure} & Greater than your skill\\
	\textbf{Success} & Less than your skill\\
	\textbf{Hard Success} & Less than half your skill\\
	\textbf{Extreme Success} & Less than one-fifth of your skill\\
\end{tabular}
\end{figure}

\begin{tcolorbox}
	\textbf{Note}: If you are making a check using a skill that is 50 or higher, only rolls 98 or higher count as a fumble.
\end{tcolorbox}

Fumbling on a roll is a critical failure. You will lose an opposed roll even if your opponent fails, and you may face abnormally bad consequences at the GM's discretion, such as being disarmed or knocked off your feet in battle. Regular failures will typically mean you simply do not accomplish your goal, on the other hand. Extreme successes will sometimes grant even more than you asked for. Achieving extreme success on an attack will do maximum damage, for instance (see \textit{Chapter 4, Combat}).\\

If a task seems like it is harder than usual, the GM may require you to make a hard or even extreme success to accomplish it. Some rules specify the conditions under which this occurs. For example, if you make a ranged attack that is greater than the range of your weapon (but less than double), it requires a hard success to hit!

\subsection{Opposed Rolls}
Whenever two characters are acting in opposition of each other, they must both make rolls to see which one performs better. Success levels are very important here; the character who achieves the higher success level wins. Ties result in no clear victor unless a tiebreaking condition is specified. For example, ties on opposed attack/defense rolls generally go to the defender (see \textit{Chapter 4, Combat}).

\subsection{Bonus and Penalty Dice}
Whenever a roll has a bonus die, reroll the percentile die and take the lower result. Remember that lower is better. Penalty dice have the opposite effect: the higher roll takes effect instead of the lower one.\\

\begin{tcolorbox}
\textbf{Example 1}: Thoronir Greenhollow wants to climb a tree to get a better view of the area. The GM decides this requires a normal Acrobatics check, meaning Thoronir must roll at or below his Acrobatics score. His Acrobatics score is 50, giving him a reasonable chance of succeeding. However, he is exhausted, giving him a penalty die on the roll. He rolls a 40 on his percentile die and a 5 on his d10, meaning his roll is a success! However, the reroll of the percentile die comes up with a 60, changing his roll to 65, meaning he failed.\\

\textbf{Example 2}: Rasha-daro is trying to sneak by a guard. The GM asks her player to make a Sneak check. It is night, and Rasha-daro's path across the room is only dimly lit. This gives her a bonus die. She needs to roll below her Sneak score of 40 to succeed. She rolls a 45, failing; however, she gets to reroll the percentile die, changing her roll to 15. She winds up with a hard success on the roll!
\end{tcolorbox}

\subsubsection{Multiple Bonuses and Penalties}
When there are multiple bonus or penalty conditions affecting a roll, they have a cumulative effect. There can be multiple dice added to the roll. For example, if you have two bonus dice on your roll, you take the best of three percentile die rolls. Wherever a roll has bonus \textit{and} penalty dice, the two cancel each other out. For example, a roll with three penalty dice and two bonus dice will wind up with only one penalty die.

\section{Non-Integer Values}
Wherever non-integer values are the final result of a calculation, always round down. For example, if your Destruction spells cost 99\% of their base cost, a spell with a base cost of 6 magicka will now have a cost of 5 magicka, even though $6*0.99=5.94$. However, intermediate values you use while completing a formula are not rounded.